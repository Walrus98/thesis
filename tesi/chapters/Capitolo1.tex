\chapter{Introduzione}

\section{Introduzione al progetto}

Il progetto di tirocinio ha mirato allo sviluppo e l'integrazione di un prototipo di applicazione per la visualizzazione di mappe geospaziali, avvalendosi degli standard dell'\textit{Open Geospatial Consortium} (\textit{OGC}) e delle più moderne tecnologie web. Esso si pone come obiettivo quello di ampliare un’applicazione web già esistente, che sia in grado di presentare dati geospaziali interattivi per l’utente che vi si interfaccia. Il lavoro svolto, della durata di circa trecento ore, si è concentrato sia sull'implementazione di un'applicazione front-end per la visualizzazione di mappe geospaziali, includendo un'interfaccia utente interattiva, sia sullo sviluppo back-end per la gestione e fornitura dei dati delle mappe. È  stato inoltre sviluppato un sistema che ha reso possibile a questi due componenti di integrarsi fra loro. 
\\Il progetto è nato dalla necessità di riuscire a visualizzare in modo più comprensibile e immediato lo stato dei ponti di tutta Italia. Esso si è appoggiato ad un'applicazione già esistente, denominata \textit{Inspicio}, sviluppata dall'azienda Geckosoft per conto del Consorzio Fabre e della facoltà di Ingegneria Civile dell'Università di Pisa. Tale applicazione è nata con l’obiettivo di creare una piattaforma completa per il censimento digitale dei ponti di tutta Italia e la valutazione della loro condizione con eventuali rischi annessi
\footnote[1]{Con censimento digitale di un ponte si intende la raccolta, organizzazione e manutenzione di tutti i dati relativi a tale ponte, con lo scopo di tenerne sotto controllo lo stato in maniera continuativa.}. 
Essa offre agli utenti la possibilità di raccogliere dati dettagliati sui ponti, come posizione geografica, dimensioni, materiali di costruzione, età e condizioni attuali. Inoltre rende consultabili informazioni relative all'ambiente circostante a queste infrastrutture censite come possibili attività sismiche, pericolo alluvioni, esondazioni di fiumi, frane, etc...
\\Il progetto dunque ha permesso di individuare con facilità la posizione dei ponti in questione, quali siano le informazioni ad essi associati e quali zone limitrofe siano interessate da rischi sismici o idrogeologici. Il problema infatti non risiedeva nella disponibilità dei dati, i quali erano già presenti all'interno dell'applicazione in forma testuale, ma nella modalità con cui venivano presentati all'utente, risultando maggiormente complicati nell'utilizzo.

\section*{Obiettivi da raggiungere}

Il progetto di tirocinio si è proposto di sviluppare un'applicazione web che consenta l'integrazione di mappe all'interno di un software preesistente. Tali mappe, indispensabili per l'utente al fine di comprendere i rischi geografici per ponti e viadotti (quali rischi idrogeologici, sismici, da frane, etc...), sono fornite da enti terzi. Esse in base all'ente fornitore vengono reperite o comunicando con server esterni, tramite l'utilizzo di una serie di protocolli di mappe standard, oppure scaricando localmente i file geospaziali sul proprio dispositivo, con la necessità di uno strumento specifico per la loro visualizzazione. 
\\L’obiettivo primario, quindi, sarebbe stato quello di sviluppare un sistema front-end capace di richiedere e interpretare questi dati, per poi renderli visibili su una mappa interattiva. Ciò avrebbe implicato, come prima fase, l'utilizzo dei protocolli precedentemente menzionati, maggiormente nello specifico il protocollo Web Map Service (WMS) e il protocollo Web Feature Service (WFS): entrambi definiti dallo standard dell'Open Geospatial Consortium (OGC). Inoltre, sarebbe stato opportuno integrare un ulteriore protocollo, noto come Web Map Tile Service (WMTS), da utilizzare (quando possibile) come alternativa del WMS, in quanto più efficiente. L'idea avrebbe previsto di estendere un componente Angular già esistente, introdotto in minima parte dall'azienda, utilizzando al suo interno una libreria denominata OpenLayers. Solo dopo aver implementato i protocolli sopracitati il lavoro si sarebbe potuto spostare sul back-end in modo da di riuscire ad introdurre un meccanismo che agisse da server di mappe. Quest'ultimo avrebbe dovuto memorizzare sia le mappe fornite dai server esterni, sia i file geospaziali disponibili solo come file locali da scaricare. Infine, avrebbe dovuto fornire i dati di queste mappe, utilizzando i protocolli sopra menzionati. Come ultima fase del progetto, il tirocinante avrebbe dovuto sviluppare un ulteriore sistema che tenesse traccia di tutte le mappe consultabili, attraverso l'uso di un database, e che fosse in grado di comunicarle al front-end mediante una Rest API, indicando quali di esse fossero disponibili per l'uso. 
\\Durante in ogni stadio del progetto, dunque, il candidato, assieme al tutor aziendale, si sarebbe dovuto occupare di valutare e scegliere le tecnologie che meglio avrebbero consentito la realizzazione di tale infrastruttura, in quanto le funzionalità legate alla gestione delle mappe era in uno stato iniziale.

\section{Struttura della relazione}

\begin{description}
    \item[{\hyperref[cap:chapter2]{Il secondo capitolo}}] si occupa di fornire una spiegazione dettagliata sul funzionamento dei vari protocolli utilizzati per reperire mappe dai servizi geospaziali.
    \item[{\hyperref[cap:chapter3]{Il terzo capitolo}}] fornisce un'analisi sulle tecnologie impiegate all'interno del progetto, così da comprendere meglio le scelte progettuali svolte durante il percorso di tirocinio.
    \item[{\hyperref[cap:chapter4]{Il quarto capitolo}}] delinea il processo di introduzione delle prime mappe all'interno del progetto, ancora prive di un servizio di back-end e analizza le problematiche riscontrate durante questa fase implementativa.
    \item[{\hyperref[cap:chapter5]{Il quinto capitolo}}] descrive le soluzioni adottate per risolvere i problemi riscontrati precedentemente, mediante l'introduzione di un nuovo servizio nel back-end. In seguito, viene fornita una spiegazione sul funzionamento di questo servizio, noto come GeoServer, che assume il ruolo di server geospaziale. Viene spiegato inoltre come questo sia stato impiegato all'interno del progetto per la gestione e il fornimento dei dati, inclusa una descrizione di come le mappe siano state implementate utilizzando questo servizio, lato front-end.
    \item[{\hyperref[cap:chapter6]{Il sesto capitolo}}] approfondisce il processo di caricamento automatico delle mappe all'interno di GeoServer, il quale è stato realizzato attraverso la creazione di un programma dedicato, situato nel back-end e denominato AutoLoader. Tale soluzione è stata implementata in risposta alla necessità di gestire un elevato volume di mappe, rendendo impraticabile il caricamento manuale a causa della complessità e del lavoro richiesto.
    \item[{\hyperref[cap:chapter7]{Il settimo capitolo}}] descrive l'integrazione tra il front-end e il server di mappe, introducendo una nuova figura che opera come REST API. Quest'ultima fornisce al client le informazioni necessarie riguardanti le mappe disponibili e stabilisce le modalità tramite le quali il client può contattare il server, utilizzando protocolli OGC diversificati in base al tipo di mappa. 
    \item[{\hyperref[cap:chapter8]{L'ottavo capitolo}}]  contiene le conclusioni che descrivono gli obiettivi raggiunti e che riassumono, in concreto, tutte le attività svolte nel percorso di tirocinio. Questo capitolo si conclude con una riflessione finale sullo stato attuale dell'applicazione realizzata.
\end{description}