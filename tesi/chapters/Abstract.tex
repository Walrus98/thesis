\cleardoublepage\phantomsection\pdfbookmark{Sommario}{Sommario}
\begingroup
\let\clearpage\relax
\let\cleardoublepage\relax
\let\cleardoublepage\relax

\chapter*{Abstract}

La presente relazione descrive il lavoro svolto durante il periodo di tirocinio formativo, della durata di circa trecento ore complessive, dal laureando Samuele Calugi presso l'azienda Geckosoft, sede di Pisa. 
\\Durante il tirocinio è stato richiesto il raggiungimento di massima dei seguenti obiettivi: lo studio autonomo degli Standard OGC (Web Map Service, Web Feature Service, Web Map Tile Service) e dei formati dati geospaziali (GeoJSON, Shapefile, ecc.); lo sviluppo di un frontend interattivo in grado di mostrare mappe ottenute attraverso gli standard OGC e i formati dati geospaziali specificati; lo sviluppo di un backend per il reperimento delle mappe in conformità agli standard OGC e ai formati dati geospaziali richiesti; l'integrazione del backend e del frontend per creare un'applicazione completa e funzionale per la visualizzazione di dati geospaziali provenienti da diverse fonti; l'utilizzo di \textit{Angular}, per la parte front-end e \textit{.NET} per la parte back-end; la documentazione dettagliata del lavoro svolto, incluse le tecnologie utilizzate, le scelte progettuali e le modalità di utilizzo.
\\A conclusione del percorso lavorativo, il candidato ha conseguito con successo tutti gli obiettivi prefissati, realizzando sia un'applicazione web che consenta agli utenti di richiedere e visualizzare un elenco di mappe disponibili, mostrando la mappa selezionata utilizzando i protocolli OGC, sia un servizio back-end che registri e pubblichi tali dati, mediante la creazione di un servizio di Rest API e un database apposito. Inoltre, il candidato ha introdotto un software dedicato che ha assunto il ruolo di server di mappe all'interno dell'applicazione, così da poter memorizzare e fornire mediante protocolli OGC i dati geospaziali. Infine lo studente si è proposto di realizzare un ulteriore programma, lato back-end, che permettesse il caricamento automatico di mappe geospaziali all'interno del server di mappe. Questa necessità è emersa per facilitare l'inserimento di tali mappe che altrimenti sarebbe stato estremamente laborioso da eseguire manualmente, vista la mole di dati.

\endgroup

\vfill